\documentclass[line]{res} 
\newenvironment{list1}{
  \begin{list}{\ding{113}}{%
      \setlength{\itemsep}{0in}
      \setlength{\parsep}{0in} \setlength{\parskip}{0in}
      \setlength{\topsep}{0in} \setlength{\partopsep}{0in} 
      \setlength{\leftmargin}{0.17in}}}{\end{list}}
\newenvironment{list2}{
  \begin{list}{$\bullet$}{%
      \setlength{\itemsep}{0in}
      \setlength{\parsep}{0in} \setlength{\parskip}{0in}
      \setlength{\topsep}{0in} \setlength{\partopsep}{0in} 
      \setlength{\leftmargin}{0.2in}}}{\end{list}}

%\hyphenpenalty=5000
%\tolerance=1000
\begin{document}

\name{Mark L. Chang}

\begin{resume}
	
	% -----------------------------------------------------------------
	% Contact info
	% -----------------------------------------------------------------
	\section{\sc Contact Information} \vspace{0.05in} 
	\begin{tabular}
		{@{}p{3in}p{2.5in}} 1000 Olin Way & {\em Mobile:} 781.559.0565 \\
		Needham, MA 02492 & {\em Email:} mark.chang@gmail.com \\
		USA & {\em Web:} http://markchang.net \\
	\end{tabular}
	
	%\section{\sc Personal}
	%Born March 11, 1975. U.S. citizenship.
	% -----------------------------------------------------------------
	% Education
	% -----------------------------------------------------------------
	\section{\sc Education} Ph.D. in Electrical Engineering, University of Washington, Seattle, WA, 2004.\\
	\vspace*{-.1in} 
	\begin{list1}
		\item[] Thesis: {\em Variable Precision Analysis for FPGA Synthesis} 
		\item[] Adviser: Scott Hauck 
	\end{list1}
	
	M.S., Electrical and Computer Engineering, Northwestern University, Evanston, IL, 2000.\\
	\vspace*{-.1in} 
	\begin{list1}
		\item[] Thesis: {\em Adaptive Computing in NASA Multi-Spectral Image Processing} 
		\item[] Adviser: Scott Hauck 
	\end{list1}
	
	B.S. with University and Departmental Honors, Electrical and Computer Engineering, The Johns Hopkins University, Baltimore, MD, 1997.
	
	\section{\sc Research Interests}
	
	Mobile, social, and ubiquitous computing; engineering education; design and student motivation; reconfigurable computing
	
	% -----------------------------------------------------------------
	% Employment
	% -----------------------------------------------------------------
	\section{\sc Employment}

	\employer{edX} 
	\title{Director of Product} 
	\dates{08/2012 - Present} \location{Cambridge, MA}
	\begin{position}
		Helping build the world's finest online learning platform.
	\end{position}
	
	\employer{Boston Startup School} 
	\title{Advisor} 
	\dates{03/2012 - Present} \location{Boston, MA}
	\begin{position}
		Designed academic program and curriculum, recruited instructors, and taught in the inaugural class of Boston Startup School. As an advisor, responsible for setting strategic vision for future programs.
	\end{position}
	
	\employer{The Boston Globe / The New York Times} 
	\title{Creative Technologist} 
	\dates{10/2011 - 08/2012} \location{Boston, MA} 
	\begin{position}
		Performed research and development of next generation news production and consumption
    technologies at The Globe Lab.
	\end{position}
	
	\employer{Franklin W. Olin College of Engineering} 
	\title{Associate Professor} 
	\dates{08/2011 - Present} \location{Needham, MA} 
	\begin{position}
		Electrical and Computer Engineering faculty member. On leave at edX beginning 08/2012.
	\end{position}
	
	\employer{Franklin W. Olin College of Engineering} 
	\title{Resident Scholar} 
	\dates{08/2005 - 09/2012} \location{Needham, MA} 
	\begin{position}
		Lived on campus as an academic resource for students at Olin College. Responsible for academic advising and intellectually stimulating activities in the residence halls.
	\end{position}
	
	\employer{Franklin W. Olin College of Engineering} 
	\title{Assistant Professor} 
	\dates{08/2004 - 08/2011} \location{Needham, MA} 
	\begin{position}
		Electrical and Computer Engineering faculty member. 
	\end{position}
	
	\employer{University of Washington} 
	\title{Graduate Research Assistant} 
	\dates{07/2000 - 07/2004} \location{Seattle, WA} 
	\begin{position}
		Developed variable precision design tools for FPGAs. 
	\end{position}
	
	\employer{Quicksilver Technologies, Inc.} 
	\title{Software Developer} 
	\dates{07/2001 - 10/2001} \location{Seattle, WA} 
	\begin{position}
		Assisted design and development of software development tools for Quicksilver's reconfigurable hardware. 
	\end{position}
	
	\employer{Northwestern University} 
	\title{Graduate Research Assistant} 
	\dates{09/1997 - 06/2000} \location{Evanston, IL} 
	\begin{position}
		Developed FPGA implementations of NASA image processing applications. 
	\end{position}
	
	\employer{National Computer Systems} 
	\title{Customer service operator} 
	\dates{06/1997 - 08/1997} \location{Iowa City, IA} 
	\begin{position}
		Phone operator for the Department of Education. 
	\end{position}
	
	\employer{Johns Hopkins University} 
	\title{Undergraduate Research Assistant} 
	\dates{10/1994 - 06/1997} \location{Baltimore, MD} 
	\begin{position}
		Worked on a portable high performance linear algebra library. Investigated the IEEE-1394 draft standard in conjunction with the JHU Applied Physics Laboratory for a 1394-based spacecraft bus design. 
	\end{position}
	
	\title{Assistant System Administrator} 
	\dates{03/1995 - 06/1997} 
	\begin{position}
		Maintained a network of servers and workstations for the Center for Language and Speech Processing. 
	\end{position}
	
	\title{Webmaster} 
	\dates{05/1995 - 01/1997} 
	\begin{position}
		Designed and maintained a web site for the Maryland Space Grant Consortium. 
	\end{position}
	
	\employer{Anton-Paar, GmbH} 
	\title{Embedded Software Developer} 
	\dates{06/1996 - 07/1996} \location{Graz, Austria} 
	\begin{position}
		Participated in a cooperative internship with the Technical University of Graz, Austria. Developed embedded software for use in concentration determination instruments. 
	\end{position}
	
	\employer{Products Unlimited, Corp.} 
	\title{Programmer and technician} 
	\dates{Summer 1989 - 1994} \location{Iowa City, IA} 
	\begin{position}
		Set up and maintained a network of PCs for a small engineering office. Developed computer-aided testing facilities using IEEE-488 instruments and hardware. 
	\end{position}
	
	% -----------------------------------------------------------------
	% Publications
	% -----------------------------------------------------------------
	\section{\sc Publications}
	
	Orit Shaer, Marina Umaschi Bers, Mark L. Chang, ``Making the Invisible Tangible: Learning Biological Engineering in Kindergarten'', \textit{Proceedings of the 2nd Workshop on UI Technologies and Their Impact on Educational Pedagogy}, May 2011.
	
	Ilari Shafer, Mark L. Chang, ``Movement Detection for Power-Efficient Smartphone WLAN Localization'', \textit{13th ACM International Conference on Modeling, Analysis and Simulation of Wireless and Mobile Systems}, October 2010.
	
	Andrew Barry, Noah Tye, Mark L. Chang, ``Interactionless Calendar-Based Training for 802.11 Localization,'' \textit{The 7th IEEE International Conference on Mobile Ad-hoc and Sensor Systems}, November 2010.
	
	Mark L. Chang, ``Work in Progress: synthesizing design, engineering, and entrepreneurship through a course in mobile application development'', \textit{Frontiers in Education Conference}, 2010.
	
	Jessica Townsend, Mark L. Chang ``Work in Progress: Impact of early design instruction on capstone experiences'', \textit{Frontiers in Education Conference}, 2010.
	
	Andrew Barry, Benjamin Fisher, Mark L. Chang, ``A Long-Duration Study of User-Trained 802.11 Localization,'' \emph{Proceedings of the Second ACM International Workshop on Mobile Entity Localization and Tracking in GPS-less Environments}, September 2009. \textit{Awarded best paper and best presentation.}
	
	Stephen Longfield, Jr., Mark L. Chang, ``A Parameterized Stereo Vision Core for \mbox{FPGAs}'', \emph{IEEE Symposium on Field-Programmable Custom Computing Machines}, April 2009.
	
	Mark L. Chang, Allen Downey, ``A Semi-Automatic Approach for Project Assignment in a Capstone Course'', \emph{Proceedings of the American Society for Engineering Education Annual Conference}, June, 2008.
	
	Mark L. Chang, Jessica Townsend, ``A Blank Slate: Creating a New Senior Engineering Capstone Experience'', \emph{Proceedings of the American Society for Engineering Education Annual Conference}, June, 2008.
	
	Mark L. Chang, ``Device Architecture'', in \emph{Reconfigurable Computing: The Theory and Practice of FPGA-Based Computation}; Scott Hauck, Andre DeHon, Editors; Morgan Kaufmann/Elsevier, 2008, pp. 3-27.
	
	Mark L. Chang, Scott Hauck, ``Pr\'{e}cis: A Design-Time Precision Analysis Tool'', \emph{IEEE Design and Test of Computers}, Vol. 22, No. 4, pp. 349-361, July-August 2005.
	
	Mark L. Chang, \emph{Variable Precision Analysis for FPGA Synthesis}, Ph.D. Dissertation, University of Washington, Department of Electrical Engineering, 2004.
	
	%put \mbox around FPGAs to prevent hyphenation
	Mark L. Chang, Scott Hauck, ``Automated Least-Significant Bit Datapath Optimization for \mbox{FPGAs}'', \emph{IEEE Symposium on Field-Programmable Custom Computing Machines}, April, 2004.
	
	Mark L. Chang, Scott Hauck, ``Variable Precision Analysis for FPGA Synthesis'', \emph{Earth Science Technology Conference}, June, 2003.
	
	Mark L. Chang, Scott Hauck, ``Pr\'{e}cis: A Design-Time Precision Analysis Tool'', \emph{Earth Science Technology Conference}, June, 2002.
	
	Mark L. Chang, Scott Hauck, ``Pr\'{e}cis: A Design-Time Precision Analysis Tool'', \emph{IEEE Symposium on Field-Programmable Custom Computing Machines}, pp. 229--238, 2002.
	
	Mark L. Chang, \emph{Adaptive Computing in NASA Multi-Spectral Image Processing}, M.S. thesis, Northwestern University, Dept. of ECE, December, 1999.
	
	Mark L. Chang, Scott Hauck, ``Adaptive Computing in NASA Multi-Spectral Image Processing'', \emph{Military and Aerospace Applications of Programmable Devices and Technologies International Conference}, 1999.
	
	P. Banerjee, A. Choudhary, S. Hauck, N. Shenoy, C. Bachmann, M. Chang, M. Haldar, P. Joisha, A. Jones, A. Kanhare, A. Nayak, S. Periyacheri, M. Walkden, ``MATCH: A MATLAB Compiler for Adaptive Computing Systems'', \emph{Northwestern University Department of Electrical and Computer Engineering Technical Report CPDC-TR-9908-013}, 1999.
	
	% -----------------------------------------------------------------
	% Conference Posters
	% -----------------------------------------------------------------
	\section{\sc Conference Posters}
	
	Mihir Ravel, Mark L. Chang, Mark McDermott, Michael Morrow, Nikola Teslic, Mihajlo Katona, Jyotsna Bapat, ``A Cross-Curriculum Open Design Platform Approach to Electronic and Computing Systems Education,'' \emph{IEEE International Conference on Microelectronic Systems Education}, July 2009. 
	
	C. Murphy, D. Lindquist, A.M. Rynning, T. Cecil, S. Leavitt, M.L. Chang, ``Low-Cost Stereo Vision on an FPGA'', IEEE Symposium on Field-Programmable Custom Computing Machines, 2007.
	
	Mark L. Chang, Scott Hauck, ``Least-Significant Bit Optimization Techniques for FPGAs'', \emph{ACM/SIGDA International Symposium on Field-Programmable Gate Arrays}, February, 2004.
	
	% -----------------------------------------------------------------
	% Panels
	% -----------------------------------------------------------------
	\section{\sc Panels and Workshops}
	
	Mark L. Chang, Jonathan Hulbert, Martha Minow, Jonathan Zittrain, ``The 2013 Hack IP Challenge'' \textit{HarvardX/edX}, February 2013.

	Hal Abelson, Mark L. Chang, Cyprien Lomas, David Wolber, ``Google App Inventor for Android: Building mobile applications as a first computing experience'' \textit{Frontiers in Education Conference}, 2010.
	
	Hal Abelson, Mark L. Chang, Eni Mustafaraj, Franklyn Turbak, ``Mobile Phone Apps in CS0 Using App Inventor for Android'', \textit{15th Annual Conference of the Northeast region of the Consortium for Computing Sciences in Colleges}, 2010.
	
	Ellen Spertus, Mark L. Chang, Paul Gestwicki, David Wolber, ``Novel Approaches to CS0 with App Inventor for Android'', \textit{The 41st ACM Technical Symposium on Computer Science Education}, 2010.
	
	\section{\sc Invited Talks}

	``Two Sides to Innovation in the Classroom'', Seoul National University, Center for Teaching and Learning, February 2013.

	``Innovation in Engineering Education: Olin College'', Seoul National University, College of Engineering, February 2013.

	``Disruption: Online Education'', Seoul National University, College of Engineering, February 2013.
	
	``Play With Others'', \textit{World Lab Summer Institute}, University of Washington Department of Computer Science and Engineering, July 2012.

	Invited speaker, Scientia Conference on Research and Innovation in Undergraduate Science and Engineering Education, Rice University, February, 2011
	
	Mark L. Chang, ``Master of motivation: engaging students with smartphones and Google Android'', \textit{Boston-area Advanced Technological Education Connections IT Futures Forum}, May 2010.
	
	``Spinning the World Wide Web: How the Internet Really Works'', Needham Exchange Club presentation, October 2008.
	
	``Olin College: Accrediting an Innovative Engineering Curriculum'', Yonsei University Engineering Seminar, August 2008.
	
	``Olin College: Rethinking Engineering Education'', Microsoft Research, June 2008
	
	``A Beginner's Guide to Bad Engineering Presentations'', University of Hartford, November 2007.
	
	``Spinning the World Wide Web: How the Internet Really Works'', Olin College Lecture Series, Needham Adult Education Program, October 2007.
	
	% -----------------------------------------------------------------
	% Donations/Grants
	% -----------------------------------------------------------------
	\section{\sc Grants}
	MIT Lincoln Laboratory, ``Active Crowdsourcing in Support of Disaster Response''.
	
	Olin Innovation Grant funding for ``Network Hacking and Cyber Security'' course development.
	
	Wellesley Tanner conference grant for work on extending the reach of the 10th anniversary Wellesley Tanner conference.
	
	\section{\sc Donations}
	
	Altera Corp., donation of FPGA hardware and software (2006-2008).
	
	AndroidCentral.com, financial support for Mobile Application Development Course (2009).
	
	Applications Technology, Inc., financial support for Mobile Application Development Course (2009).
	
	CommonsWare, textbook for Mobile Application Development Course (2009, 2010).
	
	Google, support for AppInventor curriculum development (2010).
	
	Google, hardware support for Mobile Application Development course (2011).
	
	Hewlett-Packard, Inc., donation of workstations for VLSI teaching laboratory (2005).
	
	Microsoft, hardware and software for Mobile Application Development Course (2009).
	
	Nokia Research Center, donation of handheld computing hardware (2008).
	
	Palm, Inc., donation of textbook for all students in Mobile Application Development Course (2009).
	
	Xilinx, Inc., donation of FPGA hardware and software (2004-present).
	
	% -----------------------------------------------------------------
	% Professional Activities
	% -----------------------------------------------------------------
	\section{\sc Professional Activities}
	
	Program Committee Member, Publicity Chair, IEEE Conference on Field-Programmable Custom Computing Machines, 2010.
	
	General co-chair, IEEE Workshop on Mobile Entity Localization and Tracking. Co-located with \textit{The 7th IEEE International Conference on Mobile Ad-hoc and Sensor Systems}, November 2010.
	
	Program Committee Member, IEEE Microelectronic Systems Education Conference, 2005, 2007, 2009, 2011.
	
	Program Committee Member, IEEE International Conference on Field-Programmable Technology, 2007, 2008, 2009, 2010.
	
	Program Committee Member, IEEE International Conference on Field Programmable Logic and Applications, 2005, 2006, 2007, 2008, 2009.
	
	Program Committee Member, International Symposium on Applied Reconfigurable Computing, 2008, 2009, 2010, 2011.
	
	Reviewer, IEE Proceedings of Computers \& Digital Techniques, IEEE Transactions on Computers, IEEE Transactions on Education, IEEE Transactions on VLSI Systems, IEEE Transactions on Computer-Aided Design of Integrated Circuits \& Systems, IEEE Transactions on Instrumentation \& Measurement, ACM Transactions on Design Automation of Electronic Systems, IEEE International Symposium on Circuits and Systems, EURASIP Journal of Embedded Systems, Journal of Real-Time Image Processing, ACM Transactions on Reconfigurable Technology and Systems, Hindawi International Journal of Reconfigurable Computing, ACM International Conference on Interactive Tabletops and Surfaces, ACM Symposium on User Interface Software and Technology.
	
	% -----------------------------------------------------------------
	% Consulting
	% -----------------------------------------------------------------
	\section{\sc Consulting}
	\begin{format}
		\employer{l}
		\dates{r}\\
		\body\\
	\end{format}

	\employer{MassChallenge}
	\dates{07/2012 - present}
	\begin{position}
		Mentor for various startups in the 2012 MassChallenge startup accelerator.
	\end{position}
	
	\employer{Roundware}
	\dates{12/2011 - present}
	\begin{position}
		Integration of indoor localization technologies into an open source distributed, participatory, location-aware audio platform for art installations.
	\end{position}
	
	\employer{The MITRE Corporation} 
	\dates{05/2011 - 09/2011} 
	\begin{position}
		Research investigating next-generation distributed design and collaboration tools and evaluation of mobile development frameworks.
	\end{position}
	
	\employer{SAIC (formerly Applications Technology, Inc.)} 
	\dates{06/2009-present} 
	\begin{position}
		Research supporting machine language translation software and systems. 
	\end{position}
	
	\employer{The MITRE Corporation} 
	\dates{08/2007} 
	\begin{position}
		Researcher investigating 3D virtual worlds for collaboration. 
	\end{position}
	
	\employer{NetFrameworks, Inc. \& Applied Minds, Inc.} 
	\dates{07/2001 - 09/2001} 
	\begin{position}
		Primary software developer for proprietary groupware system. 
	\end{position}
	
	\employer{Hunter Benefits Consulting Group} 
	\dates{09/2000} 
	\begin{position}
		Lead software developer. 
	\end{position}
	
	\employer{HumaniTree.com, LLC} 
	\dates{12/1998 - 03/1999} 
	\begin{position}
		Web developer and Java programmer. 
	\end{position}
	
	% -----------------------------------------------------------------
	% Committees and Department Service
	% -----------------------------------------------------------------
	\section{\sc Committees and Department Service}
	
	Academic Calendaring committee, Spring 2010 - present.\\
	Ad hoc committee on curricular innovation, Fall 2009.\\
	Committee on Diversity and the Academic Experience, 2005 - 2007.\\
	Electrical and Computer Engineering Faculty Search committee, 2004, 2005, 2007.\\
	Electrical and Computer Engineering Program Group, 2004 - present.\\
	Entrepreneurship Strategic Vision committee, Fall 2012.\\
	Faculty / IT committee, 2004 - 2007.\\
	Honor Board faculty representative, 2004 - 2009.\\
	Intercollegiate Relations Committee, 2007 - present (chair 2007 - present).\\
	Olin Certificiate Program in Engineering Studies coordinator, Fall 2008 - present.\\
	Olin College faculty liaison to Babson College and Wellesley College, Spring 2010 - present.\\
	SCOPE Director search, Fall 2009.\\
	Task force on the 2nd and 3rd year curriculum, 2007 - 2008 (chair).\\
	Wellesley Olin Working Group Committee, 2004 - 2007.\\
	
	% -----------------------------------------------------------------
	% Teaching
	% -----------------------------------------------------------------
	\section{\sc Teaching}
	
	Franklin W. Olin College of Engineering, Needham, MA\\
	\vspace{-.1in} 
	\begin{list1}
		\item[] ENGR 2210: Principles of Engineering, Fall 2010. 
		\item[] ENGR 2250: User-Oriented Collaborative Design, Spring 2011. 
		\item[] ENGR 3410: Computer Architecture, Fall 2004 - present. 
		\item[] ENGR 3430: Digital VLSI Design, Spring 2005, 2006, 2007. 
		\item[] ENGR 3426: Mixed Analog-Digital VLSI I, Fall 2007 - present 
		\item[] ENGR 3427: Mixed Analog-Digital VLSI II, Spring 2008 - present. 
		\item[] ENGR 3499a: Special Topics in Electrical and Computer Engineering, Embedded Systems Design, Spring 2007, 2008. 
		\item[] ENGR 3499a: Principles of Intelligent Systems Engineering, Spring 2009. 
		\item[] ENGR 3499a: Mobile Application Development, Spring 2009, 2010, 2011. 
		\item[] ENGR 3499b: Web Application Development, Spring 2011. 
		\item[] ENGR 4190: Senior Consulting Program for Engineering, 2005 (John Deere and Motorola), 2006 (IBM Research), 2007 (MITRE and Nortel), 2008 (MITRE), 2009 (Microsoft FUSE Labs and Linden Lab), 2010 (Autodesk and Lexmark) 
		\item[] Olin Works Co-Curricular, Fall 2005, Spring 2006. 
		\item[] Green Engineering Co-Curricular, Spring 2005. 
		\item[] Physical Security Systems Co-Curricular, Fall 2008. 
		\item[] Social Justice Reading Group, 2005 - 2009. 
	\end{list1}
	
	Yonsei University, Seoul, Korea\\
	\vspace{-.1in} 
	\begin{list1}
		\item[] Principles of Engineering, Yonsei University International Summer School, Summer 2008. 
	\end{list1}
	
	University of Washington, Seattle, WA\\
	\vspace{-.1in} 
	\begin{list1}
		\item[] EE 471: Computer Design and Organization. Instructor, Winter 2003.\\
		Overall class evaluation rating 4.13/5.0.\\
		(http://www.washington.edu/cec/e/EE471A4003.html). 
	\end{list1}
	
	Northwestern University, Evanston, IL\\
	\vspace{-.1in} 
	\begin{list1}
		\item[] B01: Introduction to Digital Logic Design. Instructor, Summer 1999.\\
		Overall class evaluation rating 5.56/6.0. 
		\item[] C91: VLSI Systems Design. Teaching Assistant, Winter 1999. 
		\item[] C92: VLSI Systems Design Projects. Teaching Assistant, Spring 1998. 
	\end{list1}
	
	\section{\sc Awards} 
	
	Intel Corporation: 2002-2003 Intel Foundation Graduate Fellowship
	
	University of Washington: Outstanding Graduate Research Assistant (2002), Nominated for the Yang Research Award (2002)
	
	Northwestern University: Royal E. Cabell Fellowship (1997), ECE Department Best Teaching Assistant Honorable Mention (1998)
	
	Johns Hopkins University: IEEE student chapter president (1996), Eta Kappa Nu chapter president (1996, 1997), Tau Beta Pi, Dean's List, Electrical and Computer Engineering Chair Award
	
	National Merit Scholar, National Computer Systems Merit Scholarship
	
\end{resume}
\end{document} 
